%Use esse arquivo para incluir novos pacotes

\usepackage[%usado para determinar medidas
top=1.78cm,
bottom=1.78cm,
left=1.65cm,
right=1.65cm,
headsep=0cm,
%showframe
]{geometry}
%\usepackage[justification=centering]{caption}
\usepackage{times}
\usepackage{enumitem}%redefinir espacos itemize
\usepackage{graphicx}
\usepackage{algorithm}
\usepackage{array}
\usepackage{url,hyperref}
\usepackage[utf8]{inputenc}
\usepackage{float}%mais controle para manipular figuras
\usepackage{caption}%manipular legenda da figura e tabela
\usepackage{mathtools}%equacoes
\usepackage[hang,flushmargin]{footmisc}
\usepackage{xcolor}
\usepackage{wrapfig} %usado para envolver figura com texto
%\usepackage[portuguese]{babel}
\usepackage{fancyhdr}%criacao do cabecalho
\usepackage{etoolbox}
\usepackage[export]{adjustbox}%mais controle para ajustar tamanho da tabela
\usepackage{comment}%ambiente para comentario
\usepackage{relsize} %usado por comandos \mathlarger
\usepackage{lipsum} 
\usepackage{algpseudocode}
\usepackage{blkarray}
\usepackage{amsmath}
\usepackage{graphicx}

\usepackage{listings}
\lstset{frame=tb,
  language=Java,
  aboveskip=3mm,
  belowskip=3mm,
  showstringspaces=false,
  columns=flexible,
  basicstyle={\small\ttfamily},
  numbers=none,
  numberstyle=\tiny\color{gray},
  keywordstyle=\color{blue},
  commentstyle=\color{dkgreen},
  stringstyle=\color{mauve},
  breaklines=true,
  breakatwhitespace=true,
  tabsize=3
}

% References
\usepackage{biblatex}
\addbibresource{references.bib}

%Idioma. Use "english" para trabalhos em inglês
\usepackage[english]{babel}