\section{Introduction}
Network Flow is a topic in graph theory that addresses the flow within a specific variant of a directed graph. The graph always has a Source node where all the flow originates and a Sink node where flow attempts to travel to. Flow is transferred from one node to the next through edges connecting them. Each edge has a capacity stating maximum amount of flow that can go through that edge; once that edge reaches its capacity, no more flow can go through it. A popular usage with Network Flow is finding the max flow, the maximum potential amount of flow from the Source to the Sink, for any graph as quickly and efficiently as possible.

Research pertaining to the max flow solution of network flow problems traces back to L.R. Ford and D. R. Fulkerson's paper published in 1962. Since then, several other algorithms have been published that are capable of computing these max flows with increasingly better run times. In 1972, Jack Edmonds and Richard Karp proposed a modification to the original Ford-Fulkerson algorithm which helped stabilize the run time and guarantee completion. Around the same time, Yefim Dinitz, proposed his own solution to max flow problem. Each of these algorithms use the augmenting paths approach to finding max flow, and these types of algorithms have not been heavily parallelized in the past. Our research attempts to discover parallelization options for these algorithms and compares them to their sequential counterparts as well as each other.