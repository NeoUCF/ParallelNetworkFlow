 \vspace{0.3cm}
%   \begin{resumo}
%      %%%%%%%%%%%%%% Não retirar este trecho
%     \noindent\footnote{Trabalho de Conclusão de Curso apresentado ao Curso de Engenharia Elétrica, Área de Ciências Exatas e Tecnológicas da Universidade do Oeste de Santa Catarina como requisito parcial à obtenção do grau de bacharel em Engenharia Elétrica.}
%      %%%%%%%%%%%%%% Não retirar este trecho
%      Este documento contém informações para a preparação da versão final do artigo científico de trabalho de conclusão de curso. Por favor siga cuidadosamente as instruções para garantir a legibilidade e uniformidade dos artigos.
%     \lipsum[1-1]
    
%   \end{resumo}
 
  
  \vspace{0.5cm}
  
  \begin{abstract}
  \noindent In graph theory, the field of Network Flow determines the max-
    imum flow possible from the source node to the sink node. In a
    network/transportation flow graph, each edge represents a flow
    and capacity. Network flow’s various real-world applications
    include infrastructure management such as roads, pipes, or elec-
    trical systems, or things more abstract such as team pairing, or
    ecology systems such as food webs. There are many algorithms
    which serve to find the maximum flow of a graph. In this pa-
    per, we explore three algorithms and we parallelize them to in-
    crease efficiency to find the max flow. The algorithms are Ford-
    Fulkerson, Edmonds-Karp, and Dinics.
  \end{abstract}
    \newline
  \begin{IEEEkeywords}
    \noindent Keywords: Network Flow, Parallel computation, Ford-Fulkerson, Edmonds-Karp, Dinics
  \end{IEEEkeywords}
  
    \vspace{0.5cm}